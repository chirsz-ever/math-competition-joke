\documentclass{exam}

\usepackage[zihao=-4, punct=kaiming, scheme=plain]{ctex}
\usepackage[margin=1in]{geometry}
\usepackage{zhnumber}
\usepackage{amsmath}
\usepackage{amssymb}
\usepackage{commath}
\usepackage[onehalfspacing, nodisplayskipstretch]{setspace}

\renewcommand{\thequestion}{\zhnum{question}}
\renewcommand{\questionlabel}{\thequestion、}
\renewcommand{\thepartno}{\arabic{numparts}} % 不重新编号 part
\renewcommand{\partlabel}{\thepartno.}

\pointname{分}
\pointformat{\heiti(本题\thepoints)}

\renewcommand{\partshook}{%
    \setlength{\leftmargin}{0pt}%
    \setlength{\itemindent}{0pt}%
}

\newcommand{\interger}{\mathbb{Z}}

\begin{document}

\begin{center}
    {\large \heiti 高中联赛一试模拟题}\\
    {\heiti(时间:80 分钟 \  满分:120)}
\end{center}

\begin{questions}

\question {\heiti(共 8 小题,每题 8 分)}

\begin{parts}
    \part 不能表示为两个素数之和的大于 $2$ 的最小偶数为 \fillin。
    \part 不能表示为 $n^2+1\ (n\in\interger^+)$ 形式的最大素数为 \fillin。
    \part 在平面上任给 $n\ (n\ge 2)$ 个点 $A_1$,$A_2$,\dots,$A_n$,
        其中任意两点间的距离不超过 $1$,则
        $\max \min\limits_{1\le i < j \le n}A_i A_j={\fillin}$。
    \part 若关于 $x$、$y$、$z$ 的方程 $\frac{4}{n}=\frac{1}{x}+
        \frac{1}{y}+\frac{1}{z}$ 无正整数解,则正整数 $n$ 的最小值为 \fillin。
    \part 记 $\sigma(n)=\sum_{\substack{d|n\\d\in\interger^+}}{d}$,则满足
        $\sigma(n)=2n$ 最小的奇数为 \fillin。
    \part 已知 $f(x)\in \interger^+[x]$,且 $f(e+\pi)=0$,则
        $\deg f(x) = \fillin$。
    \part 记 $p_n$ 表示从小到大第 $n$ 个素数,则所有满足 $p_{n+1}-p_{n}=2$
        的正整数 $n$ 的和为 \fillin。
    \part 给定正整数 $m$,若平面上任意 $n$ 个点中必有 $m$ 个点构成一个凸 $m$ 边形
        的 $m$ 个顶点,则 $n$ 的最小值为 \fillin。
\end{parts}

\question 解答题

\begin{parts}

\part[16] 证明 $\int^1_0(\sum^N_{x=1}e^{2\pi i x^k\alpha})^2
    (\sum^N_{x=1}e^{-2\pi i x^k\alpha})\dif \alpha = 0$。

\part[20] 对任意正整数 $n$,定义
\(
    f(n) =
    \linespread{1}\selectfont
    \begin{cases}
        3n+1         & \text{,当 $n$ 为奇数} \\
        \frac{n}{2} & \text{,当 $n$ 为偶数}
    \end{cases}
\)。对任意给定的正整数 $n$,试求最小的正整数 $k$,使得 $f^{(k)}(n)=1$。

\part[20] 记 $\zeta(s)=\sum^{+\infty}_{n=1}\frac{1}{n^s}$,
证明:$\zeta(s)$ 的零点除负实数外,全都具有实部 $\frac{1}{2}$。

\end{parts}

\end{questions}

\end{document}
